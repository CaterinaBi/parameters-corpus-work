\documentclass[fleqn,10pt]{wlscirep}
\usepackage[utf8]{inputenc}
\usepackage[T1]{fontenc}
\title{Parametric settings of functional projections in diachrony: Romance clefts and wh-interrogatives}

\author[1,*]{Caterina Bonan}
\author[2]{Giuseppe Samo}
% \author[1,2,+]{Christine Author}
% \author[2,+]{Derek Author}
\affil[1]{University of Cambridge, United Kingdom.}
\affil[2]{Beijing Language and Culture University, People’s Republic of China.}

\affil[*]{Corresponding author: cb2098@cam.ac.uk}

% \affil[+]{these authors contributed equally to this work}

%\keywords{Keyword1, Keyword2, Keyword3}

\begin{abstract}
This paper is an extract from a study that utilises data from ten Romance corpora to determine how the functional projections responsible for nominal clefting and for interrogative wh-movement have changed from earlier stages of Romance to present times. The data are assessed through the lens of Rizzi's (2017) parameters. Here, we present our preliminary French data.
\end{abstract}
\begin{document}

\flushbottom
\maketitle
% * <john.hammersley@gmail.com> 2015-02-09T12:07:31.197Z:
%
%  Click the title above to edit the author information and abstract
%
\thispagestyle{empty}

% \noindent Please note: Abbreviations should be introduced at the first mention in the main text – no abbreviations lists. Suggested structure of main text (not enforced) is provided below.

\section*{Introduction}

According to Rizzi’s (2017; see also Samo 2022) study of parameters, functional projections can be understood as either requiring overt movement (IM$=$1) or not (IM$=$0). This, along with other parameters such as the Spell Out of functional heads, allows a fine understanding of the way languages function, and evolve. In this study, we utilise Rizzi’s parameters to determine whether the existing understandings of cleft structure and interrogative wh-movement in Romance are tenable as is or need refining. To do so, we investigate the parametric settings of the projections involved in these structures, and their evolution over time.

\section*{Conceptual challenges}

Belletti’s (2015) cartography of clefts derives these structures utilising, depending on the nature of the structure, either Rizzi’s (1997) FocusP or Belletti’s (2004) Foc. In both cases, the involved projections require overt movement, i.e. IM=1.
FocusP has traditionally been considered responsible for the attraction of either wh-elements or contrastive foci. However, while the former is systematically attracted into SpecFocusP in Standard Italian, which suggests a setting as IM=1, the latter can surface either fronted or in situ (Bianchi 2013), rather suggesting an IM=1/0 setting. Clefts are present in the language but require certain context conditions to be met to be licensed (Larrivée XXXX). Languages like European French, on the other hand, have both shifted and in situ wh-elements (IM=1/0) but no prosodic foci, and productive clefts. The existence of Foc was originally posited to account for the existence of VS structures in Standard Italian (Belletti 2004) but many pieces of research have suggested that Italian low foci are always unmoved (Cardinaletti 2001, Samek-Lodovici 2015, Bonan 2021), thus suggesting an IM=0 setting for the language. In French, the position is not exploited in any known structure. These facts, coupled with numerous other empiric considerations, question the validity of the classification of the FPs exploited in clefts as FocusP and Foc. The functional projections investigated in this study are thus, minimally, Rizzi’s (1997), Belletti’s (2004) Foc, and the higher FPs exploited in inversed and interrogative clefts. We nonetheless also test the legitimacy of Rizzi’s (2017) splitting of FocusP into two projections surrounding IntP, as well as Cable’s (2010) understanding of wh-fronting as QP-fronting.

\section*{Working hypotheses}
The working hypotheses behind the present study are as follows:

\begin{itemize}
\item The parametrisation of functional projections evolves in the direction of no movement (IM=0 in Rizzi’s 2017 terms, see works on the diachrony of Chinese interrogatives, Aldridge 2010, or Japanese, Aldridge 2009, but also Roberts & Roussou 2003, Dadan 2019, a.o.);
\item Diachronically, one functional projection can display different settings for the same parameter at different stages. Synchronically, this ought to be disallowed;
\item When what is commonly considered as one single projection displays different parametrisations across structures (e.g., IM=1 in clefts vs IM=0 in interrogatives), the existence of two separate projections ought to be posited instead.
\end{itemize}

\section*{Methodology}

Our hypotheses have been tested utilising corpus linguistics techniques. As a preliminary assessment of the theory, we have limited the scope of the present investigation to the study of three standard Romance languages: Italian, French and European Portuguese. The corpora chosen for this study are: Archivio Datini, Corpus Epistolare Ottocentesco Digitale, Archivio del parlato italiano, and Banca dati dell’italiano parlato for Italian; Base de français medieval, Groupe d’Observation et de Recherche sur les Documents Epistolaires du Seizième siècle, ESLO 1-2, and 88milSMS for French; Tycho Brahe Parsed Corpus of Historical Portuguese, Reference Corpus of Contemporary Portuguese, and CORP-ORAL: Spontaneous Speech Corpus for European Portuguese for European Portuguese.
The data, which we classify utilising a parametrisation à la Rizzi (2017), will shed light on the diachronic evolution of the functional projections under consideration, and especially Rizzi’s (1997) FocusP and Belletti’s (2004) Foc. We shall also discuss the consequences of our innovative classification for the received cartography of clefts and the role of FocusP in the theory of interrogatives. 

\section*{Preliminary results for French}


\section*{Conclusions}


\bibliography{sample}


\section*{Acknowledgements}

Acknowledgements should be brief, and should not include thanks to anonymous referees and editors, or effusive comments. Grant or contribution numbers may be acknowledged.

\begin{figure}[ht]
\centering
\includegraphics[width=\linewidth]{stream}
\caption{Legend (350 words max). Example legend text.}
\label{fig:stream}
\end{figure}

\begin{table}[ht]
\centering
\begin{tabular}{|l|l|l|}
\hline
Condition & n & p \\
\hline
A & 5 & 0.1 \\
\hline
B & 10 & 0.01 \\
\hline
\end{tabular}
\caption{\label{tab:example}Legend (350 words max). Example legend text.}
\end{table}

Figures and tables can be referenced in LaTeX using the ref command, e.g. Figure \ref{fig:stream} and Table \ref{tab:example}.

\end{document}