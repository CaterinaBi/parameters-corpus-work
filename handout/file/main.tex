\documentclass[fleqn,10pt]{wlscirep}
\usepackage[utf8]{inputenc}
\usepackage[T1]{fontenc}
\usepackage{enumerate}
\usepackage{float} % keeps tables in the exact position they occupy in the code
\usepackage{natbib}
\usepackage{adjustbox}
\usepackage{gb4e} % leave last


\title{Parametric settings of functional projections in diachrony: Romance clefts and wh-interrogatives}

\author[1,*]{Caterina Bonan}
\author[2]{Giuseppe Samo}
% \author[1,2,+]{Christine Author}
% \author[2,+]{Derek Author}
\affil[1]{University of Cambridge, United Kingdom.}
\affil[2]{Beijing Language and Culture University, People’s Republic of China.}

\affil[*]{Corresponding author: cb2098@cam.ac.uk}

% \affil[+]{these authors contributed equally to this work}

%\keywords{Keyword1, Keyword2, Keyword3}

\begin{abstract}
This paper is an extract from a study that utilises data from ten Romance corpora to determine how the functional projections responsible for nominal clefting and for interrogative wh-movement have changed from earlier stages of Romance to present times. 
The data are assessed through the lens of \citet{rizzi2017} parameters. Here, we present our preliminary French data.
\end{abstract}
\begin{document}

\flushbottom
\maketitle
% * <john.hammersley@gmail.com> 2015-02-09T12:07:31.197Z:
%
%  Click the title above to edit the author information and abstract
%
\thispagestyle{empty}

% \noindent Please note: Abbreviations should be introduced at the first mention in the main text – no abbreviations lists. Suggested structure of main text (not enforced) is provided below.

\section*{Introduction}

\citeauthor{rizzi2017}’s (\citeyear{rizzi2017}; see also \citealt{samo2022}) study of parameters: functional projections either require overt movement (IM$=$1) or they don't (IM$=$0). 
This, along with parameters as Spell Out, allows a fine understanding of the way languages function, and evolve.\\

\noindent In this work:
\begin{itemize}
\item \vspace*{-2mm} we utilise Rizzi’s parameters to evaluate the existing understandings of cleft structure and interrogative wh-movement in Romance; 
\item \vspace*{-2mm} we investigate the parametric settings of the projections involved in these structures, and their evolution over time.
\end{itemize}

\section*{1. \citet{rizzi2017}) parameters}

\citeauthor{rizzi2017} (\citeyear{rizzi2017}: 165): Parameter: “an instruction for the triggering of a syntactic operation, expressed as a morphosyntactic feature associated to a functional head”.\\ 

\noindent\textbf{Move}: 
\begin{itemize}
    \item \vspace*{-2mm} complex operation (à la \citealt{chomsky2001});
    \item \vspace*{-2mm} involves either a head or a phrase;
    \item \vspace*{-2mm} encompasses the establishment of a probe-goal search followed by (internal) merge of the goal. 
\end{itemize}

For Rizzi, a functional head that acts as a trigger of movement may have distinct pairs of features responsible for (1) and (2):

\begin{exe}
    \ex PHRASAL MOVEMENT (\citealt{rizzi2017}: 171 (20))
        \begin{xlist}
            \ex A search feature at the phrasal level.
            \ex The corresponding internal merge feature at the phrasal level (IM) ('EPP feature').
        \end{xlist}
\end{exe}

\begin{exe}
    \ex HEAD MOVEMENT (\citealt{rizzi2017}: 171 (21)) 
        \begin{xlist}
            \ex A search feature at the lex level (Search\textsubscript{lex} Feature)
            \ex The corresponding internal merge feature, again at the lex level (IM\textsubscript{lex} Feature)
        \end{xlist}
\end{exe}

\noindent\textbf{Syntactic operations}: 
\begin{itemize}
    \item \vspace*{-2mm} simple;
    \item \vspace*{-2mm} highly learnable;
    \item \vspace*{-2mm} restricted to an extremely reduced set for reasons of learnability.
\end{itemize}	

\noindent When a functional element enters the syntax and becomes a functional head in the relevant configuration, it triggers one syntactic operation on the structure which is built. The available operations are those in (3):

\begin{exe}
    \ex SYNTACTIC OPERATIONS
        \begin{xlist}
            \ex Merge
            \ex Move
                \begin{xlist}
                    \ex Search:	Probe-goal relation at the phrasal level
                    \ex IM:	Internal merge of phrases; or:
                    \ex Search\textsubscript{lex}	Probe-goal relation at the head level
                    \ex IM\textsubscript{lex}	Internal merge of heads
                \end{xlist}
            \ex Spellout
        \end{xlist}
    \end{exe}

\noindent\textbf{Spell-out} ('null subject parameter' etc.): 
\begin{itemize}
    \item \vspace*{-2mm} deal with “variation in the obligatory, optional or impossible pronunciation of certain heads and of their immediate dependents” (\citealt{rizzi2017}: 175). 
\end{itemize}

\section*{2. Explaining language variability}

Cartography of syntactic structures (on this, see \citealt{cinquerizzi2010,rizzicinque2016}):
\begin{itemize}
    \item \vspace*{-2mm} the functional spine of human language is universal;
    \item \vspace*{-2mm} the functional spine comprises of numerous rigidly ordered functional projections;
    \item \vspace*{-2mm} \textbf{nonetheless}, languages vary to the extent in which they:
        \begin{enumerate}[i.]
            \item \vspace*{-2mm} activate the functional heads of the spine;
            \item \vspace*{-2mm} realise these projections using different strategies. 
        \end{enumerate}
\end{itemize}

\noindent FocusP (HLP) is not realised/exploited in the same way by all languages. \citet{samo2019cartography}: Focus^0 triggers movement of an XP that bears a relevant focus feature and:
\begin{itemize}
    \item \vspace*{-2mm} in languages such as Gungbe this head is phonetically realised \citep{aboh2004morphosyntax}, as in (\ref{gungbe});
    
        \begin{exe}
            \ex Gungbe (adapted from Aboh 2007: 85(9c))
                \gll [\textsubscript{FocusP}  	KÒFÍ\textsubscript{i}   [\textsubscript{Focus^0} 	wè 			[	ùn   		yró		\_\_\_\textsubscript{i}		]]]!\\
                {} Kofi        {} 			foc    {}    	1sg   	call {} {}\\
                \vspace{-3mm}
                \glt ‘I called KOFI (as opposed to, for example, Enoch)’
        \label{gungbe}
        \end{exe}

    \item \vspace*{-2mm} in languages like Italian, the head is silent (\citealt{rizzi1997fine} and related), as in (5);
        \begin{exe}
            \ex Italian (adapted from \citealt{samo2019cartography}: 146 (8))
                \gll [\textsubscript{FocusP} 	IL LIBRO\textsubscript{i}	    [\textsubscript{Focus^0} 	$\emptyset$ 		[	Gianni 		ha 		letto 	\_\_\_\textsubscript{i}	]]]!\\
                {} the book				{}		foc		{}	Gianni 		has 	read 	\_\_\_\\
                \vspace{-3mm}
                \glt ‘Gianni read THE BOOK (as opposed to, for example, the article)’	
        \end{exe}

    \item \vspace*{-2mm} in V2 languages, the head is activated by moving an already merged head, as in the German example in (6):
        \begin{exe}
            \ex	German (adapted from \citealt{samo2019cartography}: 146 (8))
                \gll [\textsubscript{SpecFoc} 	DIESES 	FRESKO 	[\textsubscript{Focus^0} 	malte				[	Giotto ]]]\\
                {} this 			fresco			{}		painted.3sg	{}	Giotto \\
                \vspace*{-3mm}
                \glt ‘Giotto painted THIS FRESCO (as opposed to, for example, the one over there)’
        \end{exe}

\end{itemize}

\noindent The variability of syntactic strategies adopted by different languages thus stems from different combinations of the syntactic operations of Merge, Move and Spell Out: 

\begin{itemize}
\item Gungbe merges FocusP and spells out Focus°; 
\item \vspace*{-2mm} Italian merges FocusP but \textbf{does not} spell out Focus^0; 
\item \vspace*{-2mm} German requires \textbf{both} head movement and phrasal movement. 
\end{itemize}

\noindent The parametrisation of the observed phenomena can be viewed as in TABLE I:

\begin{table}[H]
    \centering
    \begin{tabular}{|l|l|l|l|l|l|l|}
    \hline
     & Merge & Spell Out & Search & IM & Search\textsubscript{lex} & IM\textsubscript{lex} \\
    \hline
    Italian & 1 & 0 & 1 & 1 & 0 & 0 \\
    \hline
    Gungbe & 1 & 1 & 1 & 1 & 0 & 0 \\
    \hline
    German & 1 & 0 & 1 & 1 & 1 & 1 \\
    \hline
    \end{tabular}
    \caption{\label{tab:samp}Language variability in activating FocusP (\citealt{samo2019cartography}: 147 (10)).}
    \end{table}

\noindent The factorial combinations of the boolean operators result in fine cross-linguistic analyses of typological variations.


\section*{3. Conceptual challenges}

\citeauthor{belletti2015}’s (\citeyear{belletti2015}) cartography of clefts derives these structures utilising, depending on the nature of the structure, either \citeauthor{rizzi1997fine}’s (\citeyear{rizzi1997fine}) FocusP or \citeauthor{belletti2004}’s (\citeyear{belletti2004}) Foc. 
In both cases, the involved projections require overt movement, i.e. IM=1.

\begin{exe}
    \ex Involved functional projections diagram
\end{exe}

FocusP has traditionally been considered responsible for the attraction of either wh-elements or contrastive foci. 
However, while the former are systematically attracted into SpecFocusP in Standard Italian, which suggests a setting as IM=1, the latter can surface either fronted or in situ (\citealt{bianchi2013}), rather suggesting an IM=1/0 setting. 

\begin{exe}
    \ex Italian wh-questions
\end{exe}

\begin{exe}
    \ex Italian contrastive foci
\end{exe}

Clefts are present in the language but require certain context conditions to be met to be licensed (\citealt{larrive2022}). 

Languages like European French, on the other hand, have both shifted and in situ wh-elements (IM=1/0) but no prosodic foci, and productive clefts. 

\begin{exe}
    \ex French wh-interrogatives
\end{exe}

The different contexts in which clefts can be licansed in the two languages have been singled out by Larrivée as:
\begin{itemize}
    \item French:
    \item \vspace*{-2mm} Italian:
\end{itemize}

The existence of Foc was originally posited to account for the existence of VS structures in Standard Italian (\citealt{belletti2004}) but many pieces of research have suggested that Italian low foci are always unmoved (\citealt{cardinaletti2001,sameklodovici15,bonan21}), thus suggesting an IM=0 setting for the language. 

\begin{exe}
    \ex VS structure
\end{exe}

\begin{exe}
    \ex Low focus in Italian (no low movement)
\end{exe}

\begin{exe}
    \ex Low focus in Trevisan (shows low movement)
\end{exe}

According to \citet{bonan22}, the different parametric settings for the low Foc in Italian vs Trevisan can be understood as in Table II:

\begin{table}[ht]
    \centering
    \begin{tabular}{|l|l|l|l|l|l|l|}
    \hline
     & Merge & Spell Out & Search & IM & Search\textsubscript{lex} & IM\textsubscript{lex} \\
    \hline
    Italian & 1 & 0 & 1 & 0 & 0 & 0 \\
    \hline
    Trevisan & 1 & 0 & 1 & 1 & 0 & 0\\
    \hline
    \end{tabular}
    \caption{\label{tab:samp2}Language variability in activating FocP.}
    \end{table}

In French, the low focus position is not exploited in any known structure. 

These facts, coupled with numerous other empiric considerations, question the validity of the classification of the FPs exploited in clefts as FocusP and Foc. 

The functional projections investigated in this study are thus, minimally, \citeauthor{rizzi1997fine}’s (\citeyear{rizzi1997fine}), \citeauthor{belletti2004}’s (\citeyear{belletti2004}) Foc, and the higher FPs exploited in inversed and interrogative clefts.

\begin{exe}
    \ex 
\end{exe}

\section*{4. Working hypotheses}
The working hypotheses behind the present study are as follows:

\begin{itemize}
\item The parametrisation of functional projections evolves in the direction of no movement (IM=0 in \citeauthor{rizzi2017}’s \citeyear{rizzi2017} terms, see works on the diachrony of Chinese interrogatives \citealt{aldridge2010clause} or Japanese, \citealt{aldridge2009old}, but also \citealt{roberts2003syntactic}, \citealt{dadan2019}, a.o.).

\begin{exe}
    \ex From low movement to no movement (Archaic Chinese)
\end{exe}
\begin{exe}
    \ex From low movement to no movement (Old Japanese)
\end{exe}

\item Diachronically, one functional projection can display different settings for the same parameter at different stages;

\item When what is commonly considered as one single projection displays different parametrisations across structures (e.g., IM=1 in clefts vs IM=0 in interrogatives), the existence of two separate projections ought to be posited instead.
\end{itemize}

\section*{5. Methodology}

Our hypotheses are being tested utilising corpus linguistics techniques. 
As a preliminary assessment of the theory, we have limited the scope of the present investigation to the study of three standard Romance languages: Italian, French and European Portuguese. 

The corpora chosen for this study are: 

\begin{itemize}
\item Archivio Datini;
\item \vspace*{-2mm} Corpus Epistolare Ottocentesco Digitale;
\item \vspace*{-2mm} Archivio del parlato italiano;
\item \vspace*{-2mm} Banca dati dell’italiano parlato;
\item \vspace*{-2mm} Base de français medieval;
\item \vspace*{-2mm} Groupe d’Observation et de Recherche sur les Documents Epistolaires du Seizième siècle;
\item \vspace*{-2mm} ESLO 1-2;
\item \vspace*{-2mm} 88milSMS; 
\item \vspace*{-2mm} Tycho Brahe Parsed Corpus of Historical Portuguese;
\item \vspace*{-2mm} Reference Corpus of Contemporary Portuguese;
\item \vspace*{-2mm} CORP-ORAL: Spontaneous Speech Corpus.
\end{itemize}

The data, which we classify utilising a parametrisation à la \citet{rizzi1997fine}, are starting to shed light on the diachronic evolution of the functional projections under consideration, and especially \citeauthor{rizzi1997fine}’s (\citeyear{rizzi1997fine}) FocusP and \citeauthor{belletti2004}’s (\citeyear{belletti2004}) Foc. We shall also discuss the consequences of our innovative classification for the received cartography of clefts and the role of FocusP in the theory of interrogatives. 

\section*{6. The case of French}

\begin{itemize}
    \item interrogative strategies;
    \item ex situ-in situ alternation;
    \item regular and inversed clefts (with examples of inversed declaratives from portuguese and trevisan).
    \end{itemize}

\section*{7. Preliminary results for French}

\subsection*{From predominant ex situ to predominant in situ}

\begin{table}[H]
    \centering
    \small
    \begin{adjustbox}{width=\textwidth}
        \begin{tabular}{l|ll|ll|ll|ll|ll|ll|ll|ll}
        % \hline
        {} & \multicolumn{2}{c}{comment}  & \multicolumn{2}{c}{où} & \multicolumn{2}{c}{quand}& \multicolumn{2}{c}{quiS} & \multicolumn{2}{c}{quiO} & \multicolumn{2}{c}{quoiS} & \multicolumn{2}{c}{quoi0} & \multicolumn{2}{c}{que}\\
        \hline
        {} & EX & IN & EX & IN & EX & IN & EX & IN & EX & IN & EX & IN & EX & IN & EX & IN \\
        %\hline
        1870$-$1900 & 60 & 0 & 102 & 0 & 14 & 0 & 94 & 0 & 15 & 0 & 0 & 0 & 36 & 0 & 215 & 0 \\
        %\hline
        1900$-$1930 & 40 & 0 & 39 & 0 & 4 & 0 & 33 & 0 & 9 & 0 & 0 & 0 & 17 & 0 & 48 & 0 \\
        %\hline
        1970 (eslo 1) & 848 & 37 & 233 & 72 & EX & IN & EX & IN & EX & IN & EX & IN & EX & IN & EX & IN \\
        %\hline
        2014 (eslo 2) & 333 & 156 & 84 & 260 & EX & IN & EX & IN & EX & IN & EX & IN & EX & IN & EX & IN \\
        \hline
        \end{tabular}
    \end{adjustbox}
\caption{\label{tab:samp3}Total occurrences of non lexically-restricted wh-elements.}
\end{table}

\subsection*{From predominant head activation to predominant non-activation (i)}

\subsection*{From predominant head activation to predominant non-activation (ii)}

\subsection*{From movement to non movement in interrogative clefts}

\section*{Conclusions}


\bibliography{sample}

\section*{Acknowledgements}

\end{document}
